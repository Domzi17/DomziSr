\documentclass[a4paper,12pt]{article}
\usepackage[slovene]{babel}
\usepackage[utf8]{inputenc}
\usepackage[T1]{fontenc}
\usepackage{lmodern}
\usepackage{amsmath}
\usepackage{amsfonts}
\pagestyle{empty}
\begin{document}

\section*{Naloge iz matematike}

\begin{enumerate}


\item Dokaži, da je enačba $P \cap X \cup Q \cap X^c= \emptyset $
rešljiva natanko tedaj, ko je $Q \subseteq P^c$.

\item Pokaži:
\begin{itemize}

    \item $M=N \iff M+N=\emptyset$
    \item $M=N=\emptyset \iff M \cup N=\emptyset$

\end{itemize}


\item Ali obstaja tak izjavni izraz $A$, da bosta izraza
$(p\wedge A) \vee (p \Rightarrow \neg A)$ in $(p\Rightarrow A) \Rightarrow q$
enakovredna?

\item Dokaži:
\begin{itemize}
    \item $(A \Rightarrow B) \sim (\neg B \Rightarrow \neg A)$
    \item $\neg (A\vee B)\sim \neg A \wedge \neg B$
\end{itemize}


\item Poišči preneksno obliko formule $\exists x: P(x) \wedge \forall x: Q(x)\Rightarrow \forall x: R(x)$.

\item Vektorja $\vec{c}=\vec{a}+2\vec{b}$ in $\vec{d}=\vec{a}-\vec{b}$
sta pravokotna in imata dolžino 1. Določi kot med vektorjema $\vec{a}$ in $\vec{b}$.

\item Določi definicijsko območje funkcije
\[
f(x)=\log \frac{x^2 +1}{x^2 -4x+3}
\]

\item Izračunaj
\[
    \cos^2{\frac{3\pi}{8}}+\cos^2{\frac{5\pi}{8}}+\cos^2{\frac{7\pi}{8}}+\cos^2{\frac{8\pi}{8}}
\]

\item Dokaži, da za vsa naravna števila $n$ velja
\[
    \frac{1}{\sqrt{1}}+\frac{1}{\sqrt{2}}+\dots+\frac{1}{\sqrt{n}}\geq \sqrt{n}
\]

\item Naj bo $z$ kompleksno število, $z\neq1$ in $|z|=1$.
Dokaži, da je število $i=\frac{z+1}{z-1}$ realno.

\item Pokaži, da je funkcija $x\mapsto \sqrt{x}$ enakomerno zvezna na $[0,\infty)$.

\item Izračunaj limito
\[
    \lim_{x\rightarrow\infty}(\sin{\sqrt{x+1}}-\sin{\sqrt{x}})
\]

\item Dani sta grupi $(G,*)$ in $(H,\circ)$. V množici $G \times H$ definiramo operacijo
\[
    (g_1,h_1)\cdot (g_2,h_2)= (g_1*g_2,h_1 \circ h_2)
\]
Pokaži, da je množica $G \times H$ grupa za to operacijo.

\item Pokaži, da ima $f(x)=3x+\sin(2x)$ inverzno funkcijo in izračunaj $(f^-1)'(3\pi)$.

\item Izračunaj integral korenske funkcije
\[
    \int{\frac{2+\sqrt{x+1}}{(x+1)^2 -\sqrt{x-1}}}dx
\]
\item Krivulja je podana parametrično z enačbama
\[
    x(t)=\int_{t}^{1} \frac{\sin u}{u^2}du \qquad y(t)=\int_{t}^{1} \frac{\cos u}{u^2}du 
\]
Izračunaj dolžino poti od točke $x=0$ do točke, v kateri je tangenta prvič navpična.

\item Naj bo $\sum_{n=1}^{\infty}a_n$ absolutno konvergentna vrsta in $a_n\neq 1$ za $n\in \mathbb{N}$.
Dokaži, da sta vrsti
??
absolutno konvergentni.

Funkcijsko zaporedje ?? enakomerno konvergira na ?? proti funkciji ??.
Naj bo ?? zvezna. Dokaži, da funkcijsko zaporedje ??
enakomerno konvergina na ?? in določi njegovo limitno funkcijo.

Izračunaj limito zaporedja
??

Izračunaj
??

Poenostavi
??

Za dani zaporedji preveri, ali sta konvergentni
??

Ugotovi, ali obstaja
??
Pomagaj si z limitama funkcije ?? v ?? in ??.

Izračunaj naslednjo determinanto ??, ki ima na neoznačenih mestih ničle.
??

Dana je funkcija
??
Določi parameter ?? tako, da bo ?? zvezna.
Izračunaj parcialna odvoda ?? in ?? za ??.
Izračunaj parcialna odvoda ?? in ??.
Če obstaja, izračunaj limito
??
Ali je funkcija ?? diferenciabilna?

Poišči vse rešitve enačbe
??

Dokaži binomsko formulo: za vsaki realni števili ?? in ?? in za vsako naravno število ?? velja
??

Naj bo
??
Pokaži, da je ?? podgrupa v grupi ??
neničelnih kompleksnih števil za običajno množenje.
Pokaži, da je ?? podgrupa v aditivni grupi ??
ravninskih vektorjev za običajno seštevanje po komponentah.
Pokaži, da je preslikava ??, podana s pravilom
??
izomorfizem grup ?? in ??.

Nariši grafe funkcij:
??

\end{enumerate}

\end{document}
